\documentclass[12pt]{article}

\usepackage[NoDate]{currvita}
\usepackage[letterpaper, margin=0in]{geometry}
\usepackage{hyperref}
\usepackage{fontspec}
\usepackage[svgnames]{xcolor}

\renewcommand{\cvheadingfont}{\rmfamily\bfseries\Huge}
\renewcommand{\cvlistheadingfont}{\rmfamily\scshape\Large}
\renewcommand{\familydefault}{\sfdefault}
\hypersetup{colorlinks,urlcolor=Crimson}

\addtolength{\hoffset}{0in}
\addtolength{\voffset}{0.125in}

\setsansfont{Roboto}[
    Path=./fonts/Roboto/,
    Extension = .ttf,
    UprightFont=*-Light,
    BoldFont=*-Bold,
    ItalicFont=*-Italic,
    BoldItalicFont=*-BoldItalic
  ]

\setromanfont{Montserrat}[
    Path=./fonts/Montserrat/,
    Extension = .ttf,
    UprightFont=*-Thin,
    BoldFont=*-Light,
    ItalicFont=*-Italic,
    BoldItalicFont=*-BoldItalic
  ]

\begin{document}

\centering
\begin{cv}{Luis Mart\'inez}
  
  \vspace{0.25em}
  \textcolor{gray}{Telecom Engineer} •
  \textcolor{gray}{Software Engineer} •
  \textcolor{gray}{DevOps}

  \vspace{0.125em}
  \href{mailto:luis@luisalejandro.org}{luis@luisalejandro.org} •
  \href{https://luisalejandro.org/portfolio}{luisalejandro.org/portfolio}
  \vspace{0.125em}

  \hrulefill

  \begin{minipage}[t]{0.35\textwidth}
    \vspace{0.125em}
    
    \begin{minipage}{\linewidth}
      \textrm{\textsc{\Large{Educaación}}}
      \newline
      \parbox[t]{\linewidth}{
        \textbf{UNEFA} • \textrm{\textbf{Telecom Engineer}}\\
        2009 • Maracay, Venezuela\\
        \footnotesize{\textcolor{gray}{Diseño, desarrollo, implementación y depuración de sistemas de comunicaciones eléctricos, electrónicos, electromagnéticos u ópticos.}}\\
      }
      \newline
      \parbox[t]{\linewidth}{
        \textbf{CNTI} • \textrm{\textbf{Python avanzado}}\\
        2011 • Caracas, Venezuela\\
        \footnotesize{\textcolor{gray}{Estructuras, entornos virtuales, sistemas de control de versiones, reStructuredText, Sphinx, Django, Sqlite, GUI (pyGTK, pyQT), iteradores, generadores, decoradores, pruebas unitarias.}}\\
      }
      \newline
      \parbox[t]{\linewidth}{
        \textbf{CNTI} • \textrm{\textbf{Python básico}}\\
        2011 • Caracas, Venezuela\\
        \footnotesize{\textcolor{gray}{Intérprete, biblioteca estándar, funciones básicas, tipos de datos, módulos, ciclos, interacción con el usuario y el sistema operativo, formato de salida, manejo de errores.}}\\
      }
      \newline
    \end{minipage}

    \vspace{0.25em}
    \begin{minipage}{\linewidth}
      \textrm{\textsc{\Large{Skills}}}
      \newline
      \parbox[t]{\linewidth}{
        \textbf{Languages} • \footnotesize{\textcolor{gray}{Español (nativo), Inglés (fluido)}}
      }
      \parbox[t]{\linewidth}{
        \textbf{Programming} • \footnotesize{\textcolor{gray}{Node, Python, Golang, PHP, Shell script, Make.}}
      }
      \parbox[t]{\linewidth}{
        \textbf{Frontend} • \footnotesize{\textcolor{gray}{Ionic, React, NextJS, Jekyll, Django, Flask.}}
      }
      \parbox[t]{\linewidth}{
        \textbf{Backend} • \footnotesize{\textcolor{gray}{NestJS, Express, TypeORM, Firebase, Elastic, MySQL, MongoDB, Redis.}}
      }
      \parbox[t]{\linewidth}{
        \textbf{DevOps} • \footnotesize{\textcolor{gray}{Docker, Kubernetes, Git, AWS, GCP, Github Actions, Ansible.}}
      }
      \parbox[t]{\linewidth}{
        \textbf{Diagramming} • \footnotesize{\textcolor{gray}{Markdown, reStructuredText, \LaTeX.}}
      }
      \parbox[t]{\linewidth}{
        \textbf{Other} • \footnotesize{\textcolor{gray}{Empaquetado de Debian, kernels personalizados e imágenes ISO instalables para distribuciones GNU/Linux. Empaquetado de Python.}}
      }
      \newline
      \newline
    \end{minipage}

    \vspace{0.25em}
    \begin{minipage}{\linewidth}
      \textrm{\textsc{\Large{Conferencias}}}
      \newline
      \parbox[t]{\linewidth}{
        \textbf{Debian Day} • \textrm{\textbf{Orador}}\\
        2018 • Maracay, Venezuela\\
        \footnotesize{\textcolor{gray}{El mundo interno del programador.}}\\
      }
      \newline
      \parbox[t]{\linewidth}{
        \textbf{DebConf 2012} • \textrm{\textbf{Orador/Desarrollador}}\\
        2012 • Managua, Nicaragua\\
        \footnotesize{\textcolor{gray}{Haciendo distribuciones derivadas con Canaima Semilla.}}\\
      }
      \newline
    \end{minipage}

    \vspace{0.25em}
    \begin{minipage}{\linewidth}
      \textrm{\textsc{\Large{Proyectos}}}
      \newline
      \parbox[t]{\linewidth}{
        \textbf{Dockershelf} • \href{https://github.com/Dockershelf/dockershelf}{Dockershelf/dockershelf}\\
      }
      \parbox[t]{\linewidth}{
        \textbf{Agoras} • \href{https://github.com/LuisAlejandro/agoras}{LuisAlejandro/agoras}\\
      }
      \parbox[t]{\linewidth}{
        \textbf{Spices} • \href{https://github.com/LuisAlejandro/spices}{LuisAlejandro/spices}\\
      }
    \end{minipage}
  
  \end{minipage}\hspace{0.5cm}
  \begin{minipage}[t]{0.55\textwidth}
    \vspace{0.125em}

    \textrm{\textsc{\Large{Experience}}}
    \newline
    \parbox[t]{\linewidth}{
      \textbf{\href{https://wheeltheworld.com}{Wheel The World}} • \textrm{\textbf{Senior Software Engineer}}\\
      Jun 2022 --- Jul 2023 • Berkeley, United States (Remoto)\\
      \footnotesize{\textcolor{gray}{Trabajé agregando módulos CRUD a aplicaciones basadas en Next.js y TypeORM. Automaticé procesos de carga de datos usando Golang. Desarrollé el esquema CI/CD de la compañía usando Docker, Google Cloud Run y Github Actions. Implementé la integración de precios y booking con expedia.com. Desarrollé modulos para una aplicación de gestión de contenidos basada en October CMS (Laravel, PHP).}}\\
    }\vspace{0.125em}
    \parbox[t]{\linewidth}{
      \textbf{\href{https://collagelabs.org}{Collage Labs}} • \textrm{\textbf{Fundador}}\\
      Jan 2020 --- Jan 2023 • Maracay, Venezuela\\
      \footnotesize{\textcolor{gray}{Trabajé con analistas de requerimientos y desarrolladores supervisando el cumplimiento de tareas para clientes en todo el mundo. Trabajé con diseñadores para definir y lanzar el sitio web y la presencia en las redes sociales. Identifiqué y maximicé oportunidades de venta resultando en 6 nuevos clientes.}}\\
    }\vspace{0.125em}
    \parbox[t]{\linewidth}{
      \textbf{\href{https://soleit.app}{Soleit}} • \textrm{\textbf{Full Stack Developer}}\\
      Oct 2020 --- Jun 2022 • Santiago, Chile (Remoto)\\
      \footnotesize{\textcolor{gray}{Desarrollé el backend y el frontend de SoleTech 2.0, una innovadora aplicación de escritorio que permite diagnosticar y diseñar plantillas especializadas para pies. Implementé un contenedor Electron para una aplicación Ionic-React. Usé Express.js, Sequelize y Postgres para crear una API REST. Su lanzamiento permitió a Soleit entrar en la etapa de crecimiento del ciclo de vida de una startup, con financiamiento de Ganesha Lab, UC Davis y CORFO Chile.}}\\
    }\vspace{0.125em}
    \parbox[t]{\linewidth}{
      \textbf{\href{https://guayoyo.io}{Guayoyo}} • \textrm{\textbf{Python Developer}}\\
      Jun 2019 --- Dec 2021 • Montevideo, Uruguay (Remoto)\\
      \footnotesize{\textcolor{gray}{Se implementaron optimizaciones de velocidad y memoria para el \href{https://howlermonkey.io}{HowlerMonkey} updater, lo que se tradujo en una reducción de memoria del 66\% y actualizaciones más rápidas. Desarrollé módulos para agregar fuentes de vulnerabilidades al índice, incluyendo: OpenVAS, Nessus, Metasploit, ExploitDB, PacketStorm, Snort, Suricata y Nmap, permitiendo a Guayoyo agregar valor, aumentar ventas y reducir costos.}}\\
    }\vspace{0.125em}
    \parbox[t]{\linewidth}{
      \textbf{\href{https://web.archive.org/web/20180413143616/https://webuzz.es/}{Webuzz}} • \textrm{\textbf{Wordpress Developer}}\\
      Apr 2019 --- Apr 2020 • Madrid, Spain (Remoto)\\
      \footnotesize{\textcolor{gray}{Desarrollé los sitios web de \href{https://web.archive.org/web/20201101021046/https://www.recetags.com/}{Recetags} y \href{https://web.archive.org/web/20191206145303/https://www.ginburdon.com/}{Gin Burdon} usando tecnologías Wordpress y AJAX.}}\\
    }\vspace{0.125em}
    \parbox[t]{\linewidth}{
      \textbf{\href{https://leadboxhq.com}{Leadbox}} • \textrm{\textbf{Full Stack Developer}}\\
      May 2018 --- Feb 2020 • Ottawa, Canada (Remoto)\\
      \footnotesize{\textcolor{gray}{Implementé un cola de compilación de assets basado en Webpack y una arquitectura cliente-servidor. Desarrollé Marvin, una aplicación de creación de videos configurable por archivos json. Desarrollé un sistema de puesta en producción para sitios Wordpress utilizando Ansible. Se implementaron optimizaciones de UI/UX en Leaderboard, una aplicación web Angular. Creé plugins de Wordpress. Desarrollé la página web de \href{https://leadboxhq.com}{Leadbox}.}}\\
    }\vspace{0.125em}
    \parbox[t]{\linewidth}{
      \textbf{\href{https://guayoyo.io}{Guayoyo}} • \textrm{\textbf{Python Developer}}\\
      Jul 2017 --- May 2018 • Montevideo, Uruguay (Remoto)\\
      \footnotesize{\textcolor{gray}{Creé el backend para \href{https://howlermonkey.io}{HowlerMonkey}, una aplicación de informes de vulnerabilidad. Implementé la API REST de HowlerMonkey usando Flask.}}\\
    }\vspace{0.125em}
    \parbox[t]{\linewidth}{
      \textbf{\href{https://www.cnti.gob.ve}{CNTI}} • \textrm{\textbf{Linux Architect}}\\
      Nov 2009 --- Jul 2014 • Caracas, Venezuela\\
      \footnotesize{\textcolor{gray}{Desarrollé las versiones 3.0, 3.1, 4.0 y 4.1 del sistema operativo Canaima GNU/Linux, lo que permitió al gobierno de Venezuela reducir los costos de producción de 6 millones de computadoras portátiles para el programa \textsl{Canaima Educativo}.}}\\
    }
  \end{minipage}

\end{cv}
\end{document}
